\documentclass{beamer}
\usepackage[utf8]{inputenc}
%\usepackage[ngerman]{babel}

\usepackage{graphicx}
\usepackage{booktabs}
\usetheme{tud}
\usefonttheme{tud}
\usepackage{tikz}
\usetikzlibrary{automata}
\usetikzlibrary{positioning}
\usetikzlibrary{fit}
\usepackage{tikz-uml}
\usepackage{scalefnt}
\usepackage{hyperref}
\usepackage{pgfplots}
\usepackage{csquotes}

\title{Integration and Evaluation of an ASP-Solver as an Alternative
Reasoning Backend in the Rulewerk Toolkit}
\author{Steffen Breuer}
\institute[Knowledge Based Systems]{TU Dresden}






\begin{document}
\maketitle

%\begin{frame}
%\frametitle{Content}
%\tableofcontents
%\end{frame}

\section{Introduction}
\subsection{Introduction}
\begin{frame}
\frametitle{Introduction}
\begin{itemize}
\item What is Rulewerk?
\item What is the current backend?
\item Why an alternative reasoning backend?
\item What different backend was choosen?

\end{itemize}
\end{frame}

\begin{frame}
\frametitle{Rulewerk}
\begin{itemize}
\item Java API
\item Knowledge modelling, data integration, and declarative computing
\item \textbf{VLog}
\begin{itemize}
\item Datalog with existential rules and stratified negation
\item C++
\end{itemize}
\item CSV, DLGP format (Graal), define in Java, OWL ontologies, RDF, SPARQL query language
\end{itemize}
\end{frame}

\begin{frame}
\frametitle{Clingo}
\begin{itemize}
\item Answer Set Programming (ASP) system
\item gringo, clasp
\item C++, C, Python
\end{itemize}

\end{frame}

\section{Integration}
\subsection{General Workflow}

\begin{frame}
\frametitle{Gerneral Workflow}

\begin{tikzpicture}
\begin{umlseqdiag}
\umlobject [x=0] {Rulewerk}
\umlobject [x=2.5] {JNI-Java}
\umlobject [x=5] {JNI-C++}
\umlobject [x=7.5] {Control}
\umlobject [x=10]{Clingo}

\begin{umlcall}[op={translateData}]{Rulewerk}{Rulewerk}
\end{umlcall}
\begin{umlcall}[op={addData}]{Rulewerk}{JNI-Java}
\begin{umlcall}[op={addData}]{JNI-Java}{JNI-C++}
\begin{umlcall}[op={addData}]{JNI-C++}{Control}
\begin{umlcall}[op={addData}]{Control}{Clingo}
\end{umlcall}
\end{umlcall}
\end{umlcall}
\end{umlcall}

\end{umlseqdiag}

\end{tikzpicture}

\end{frame}

\begin{frame}
\frametitle{Gerneral Workflow}

\begin{tikzpicture}
\begin{umlseqdiag}
\umlobject [x=0] {Rulewerk}
\umlobject [x=2.5] {JNI-Java}
\umlobject [x=5] {JNI-C++}
\umlobject [x=7.5] {Control}
\umlobject [x=10]{Clingo}
\begin{umlcall}[op={reason}]{Rulewerk}{JNI-Java}
\begin{umlcall}[op={reason}]{JNI-Java}{JNI-C++}
\begin{umlcall}[op={reason}]{JNI-C++}{Control}
\begin{umlcall}[op={reason}]{Control}{Clingo}
\end{umlcall}
\end{umlcall}
\end{umlcall}
\end{umlcall}

\end{umlseqdiag}

\end{tikzpicture}

\end{frame}

\begin{frame}
\frametitle{Gerneral Workflow}
{\scalefont{0.85}
\begin{tikzpicture}
\begin{umlseqdiag}
\umlobject [x=0] {Rulewerk}
\umlobject [x=2.5] {JNI-Java}
\umlobject [x=5] {JNI-C++}
\umlobject [x=7.5] {Control}
\umlobject [x=10]{Clingo}

\begin{umlcall}[op={translateQuery}]{Rulewerk}{Rulewerk}
\end{umlcall}

\begin{umlcall}[op={query}, return=terms]{Rulewerk}{JNI-Java}
\begin{umlcall}[op={query}, return=terms]{JNI-Java}{JNI-C++}
\begin{umlcall}[op={query}, return=terms]{JNI-C++}{Control}
\begin{umlcall}[op={query}, return=terms]{Control}{Clingo}
\end{umlcall}
\end{umlcall}
\end{umlcall}
\end{umlcall}

\begin{umlcall}[op={makeFact}]{Rulewerk}{Rulewerk}
\end{umlcall}
\end{umlseqdiag}

\end{tikzpicture}
}
\end{frame}

\begin{frame}
\frametitle{General Workflow}

\begin{tikzpicture}
\begin{umlseqdiag}
\umlobject[x=0] {Rulewerk}
\umlobject[x=10] {VLog}

\begin{umlcall}[op={loadInMemoryDatasource}]{Rulewerk}{VLog}
\end{umlcall}

\begin{umlcall}[op={query}, return=facts]{Rulewerk}{VLog}
\end{umlcall}

\begin{umlcall}[op={safeFactsToKB}]{Rulewerk}{Rulewerk}
\end{umlcall}
\end{umlseqdiag}

\end{tikzpicture}

\end{frame}

\subsection{Interaction with Clingo}
\begin{frame}
\frametitle{Interacting with Clingo}
\begin{itemize}
\item How to add data?
\item How to reason?
\item How to get data out of Clingo?
\item Why abstract the control of Clingo?
\end{itemize}
\end{frame}

\begin{frame}
\frametitle{Add Data}
ControlClingo
\begin{itemize}
\item Rules and facts (statements)
\item Parse into Clingo syntax myself
\begin{itemize}
\item only addable after grounding
\end{itemize}
\item Add statements as strings
\item Configurations % cause for abstraction
\begin{itemize} 
\item setting a part
\item setting parameters
\end{itemize}

\end{itemize}
\end{frame}

\begin{frame}
\frametitle{Reason over a Knowledge Base}
ControlClingo
\begin{itemize}
\item ground
\begin{itemize}
\item set part
\item set params
\end{itemize}
\item solve
\begin{itemize}
\item add additional facts
\item save model as SymbolSpan (Vector)
\end{itemize}
\end{itemize}
\end{frame}

\begin{frame}
\frametitle{Get a model out of Clingo}
QueryTermIterator
\begin{itemize}
\item Returns the terms of a specific predicate/query
\item First filtering SymbolSpan
\item Filters SymbolSpan on the fly
\end{itemize}

\end{frame}

\subsection{From Java to C++}
\begin{frame}
\frametitle{From Java to C++}
Java Native Interface
\begin{itemize}
\item Native methods
\item Conversion of objects
\end{itemize}
\end{frame}

\subsection{Usage within Rulewerk}
\begin{frame}
\frametitle{Usage within Rulewerk}
\begin{itemize}
\item Translate Rulewerks syntax into Clingos syntax
\item Make Clingos functions accessible
\begin{itemize}
\item ClingoReasoner
\end{itemize}
\item load CSV files
\begin{itemize}
\item CSVloader
\end{itemize}
\item Query a resulting model
\begin{itemize}
\item ClingoQueryResultIterator
\end{itemize}
\end{itemize}
\end{frame}

\subsubsection{Translation}
\begin{frame}
\frametitle{Rulewerk vs Clingo syntax}
\begin{itemize}
\item Both use rules and facts
\item Both support negation
\item Clingo does not support existential rules
\item Rulewerk uses \enquote{?} / \enquote{!} in front of variables
\item Rulewerk allows variables starting with a lower case letter
\item Clingo does not support IRIs

\end{itemize}

\end{frame}



\begin{frame}
\frametitle{Translation: IRIs and Variables}
\begin{itemize}
\item IRIs
\begin{itemize}
\item Introduce aliases
\item Needs inner state
\end{itemize}
\item Variables
\begin{itemize}
\item make sure variables are in upper case
\item remove \enquote{?}/\enquote{!}
\end{itemize}
\end{itemize}


\end{frame}

\begin{frame}
\frametitle{Translation: Existential Rules}
\begin{itemize}
\item Skolemization
\end{itemize}
$ h(!X) \vdash b_1(?Y),b_2(?Z). \rightarrow h(f_1(Y,Z)) \vdash b_1(Y),b_2(Z).$
\begin{itemize}
\item Needs inner state
\end{itemize}

\end{frame}

\begin{frame}
\frametitle{Translation: Implementation}
\begin{tikzpicture}
\begin{umlpackage}{Translation}

\umlclass[x=0,y=3.5]{ClingoReasoner}{}{}
\umlclass[x=0,y=0]{AliasHandler}{}{}
\umlclass[x=3,y=0]{Skolemizer}{}{}
\umlclass[x=7,y=2]{ClingoToModelConverter}{}{}
\umlclass[x=7,y=3.5]{ModelToClingoConverter}{}{}
\umluniaggreg[mult2=1]{AliasHandler}{ClingoReasoner}
\umluniaggreg[mult2=1]{Skolemizer}{ClingoReasoner}
\end{umlpackage}
\end{tikzpicture}
\end{frame}

\section{Evaluation}
\begin{frame}
\frametitle{Setup}
\begin{itemize}
\item Lenovo ideapad 520S, 8 GB of RAM, Intel Core i5-72000U 
\item Dataset: Lehigh University Benchmark (LUBM)
\item OWL lite
\end{itemize}
\begin{table}
\centering
\begin{tabular}{|c|c|c|}
\hline
\# Rules & \# Datalog rules & \# Existential Rules \\
\hline
\hline
96 & 88 & 8 \\
\hline
\end{tabular}
\caption{TBox specs}
%\lable{tab:my_lable}
\end{table}

\begin{table}
\centering
\begin{tabular}{|c|c|c|c|}
\hline
ABox & Classes & Properties & facts \\
\hline
\hline
1 & 14 & 12 & 2283599 \\
2 & 14 & 12 & 4549977 \\
\hline
\end{tabular}
\caption{ABox specs}
%\lable{tab:my_lable}
\end{table}

\end{frame}


%\pgfplotstableread[row sep=\\,col sep=&]{
%    interval & carT & carD & carR \\
%   whole run     & 1.2  & 0.1  & 0.2  \\
%   loading from File  & 12.8 & 3.8  & 4.9  \\
%   loading into Clingo    & 15.5 & 10.4 & 13.4 \\
%   solving   & 14.0 & 17.3 & 22.2 \\
%    }\mydata
    
%\begin{frame}
%\frametitle{Results}
%\begin{tikzpicture}
%\begin{axis}[ybar,
%			symbolic x coords={whole run,loading from File,loading into Clingo, solving},
%			xtick=data,
%			]
%			\addplot table[x=interval,y=carT]{\mydata};
%			\end{axis}

%\end{tikzpicture}
%\end{frame}

\begin{frame}
\frametitle{Results}
{\scalefont{0.75}
\begin{table}
\center
\begin{tabular}{|c|c|c|c|c|c|}
\hline
Setup & whole run & loading from File & loading into backend & solving & querying \\
\hline
\hline
Clingo & 87.892 & 16.201 & 25.708 & 23.490 & 22.469 \\
VLog 1 & 30.942 & 15.324 & 8.388 & 1.301 & 5.516 \\
VLog 2 & 12.810 &        & 6.737 & 1.063 & 4.801 \\
\hline

\end{tabular}
\caption{times in seconds for dataset 1}
\end{table}
}

{\scalefont{0.75}
\begin{table}
\center
\begin{tabular}{|c|c|c|c|c|c|}
\hline
Setup & whole run & loading from File & loading into backend & solving & querying \\
\hline
\hline
Clingo & 194.966 & 29.561 & 61.256 & 55.392 & 47.410 \\
VLog 1 & 106.950 & 31.156 & 58.021 & 2.945 & 14.386 \\
VLog 2 & 25.606  &        & 12.800 & 2.102 & 10.238 \\
\hline

\end{tabular}
\caption{times in seconds for dataset 2}
\end{table}
}
\end{frame}

\begin{frame}
\frametitle{Conclusion}
\begin{itemize}
\item It is possible to integrate Clingo
\item Skolemization needed to express existential rules
\item IRIS need to be replaced
\item CSV files need to be passed through VLog
\item Clingo is slow
\end{itemize}
\end{frame}

\begin{frame}
\frametitle{Improvements}
\begin{itemize}
\item Improve skolemization by adding a function to access existential/datalog variables for predicates
\item Load CSV files direktly into Rulewerk
\item use the C API of Clingo
\end{itemize}
\end{frame}
%\begin{frame}
%\frametitle{Sources}
%\url{https://iccl.inf.tu-dresden.de/web/Rules_Tutorial_2020}

%\end{frame}


\end{document}